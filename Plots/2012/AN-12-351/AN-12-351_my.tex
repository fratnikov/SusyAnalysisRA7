% Customizable fields and text areas start with % >> below.
% Lines starting with the comment character (%) are normally removed before release outside the collaboration, but not those comments ending lines

% svn info. These are modified by svn at checkout time.
% The last version of these macros found before the maketitle will be the one on the front page,
% so only the main file is tracked.
% Do not edit by hand!
\RCS$Revision: 148749 $
\RCS$HeadURL: svn+ssh://lshchuts@svn.cern.ch/reps/tdr2/notes/AN-12-351/trunk/AN-12-351.tex $
\RCS$Id: AN-12-351.tex 148749 2012-09-22 14:38:18Z alverson $

%%%%%%%%%%%%% local definitions %%%%%%%%%%%%%%%%%%%%%
\newcommand{\fullLumi}{9.2~\textrm{fb}^{-1}}
\newcommand{\procLumi}{9.2~\textrm{fb}^{-1}}
\newcommand{\iso}{\mathrm{iso}}
\newcommand{\noniso}{\mathrm{noniso}}
\newcommand{\eiso}{\epsilon_\iso}
\newcommand{\eisomu}{\eiso(\mu)}
\newcommand{\eisotr}{\eiso(\mathrm{track})}
\newcommand{\esb}{\epsilon_\mathrm{sb}}
%\newcommand{\MET}{\ensuremath{\displaystyle{\not} E_{\rm T}}}
\newcommand{\TTbar}{\ensuremath{\rm t\bar{t}}}
%%%%%%%%%%%%%%%%%%%%%%%%%%%%%%%%%%%%%%%%%%%

% This allows for switching between one column and two column (cms@external) layouts
% The widths should  be modified for your particular figures. You'll need additional copies if you have more than one standard figure size.
\newlength\cmsFigWidth
\ifthenelse{\boolean{cms@external}}{\setlength\cmsFigWidth{0.85\columnwidth}}{\setlength\cmsFigWidth{0.4\textwidth}}
\ifthenelse{\boolean{cms@external}}{\providecommand{\cmsLeft}{top}}{\providecommand{\cmsLeft}{left}}
\ifthenelse{\boolean{cms@external}}{\providecommand{\cmsRight}{bottom}}{\providecommand{\cmsRight}{right}}
%%%%%%%%%%%%%%%  Title page %%%%%%%%%%%%%%%%%%%%%%%%
\cmsNoteHeader{AN-12-351} % This is over-written in the CMS environment: useful as preprint no. for export versions
% >> Title: please make sure that the non-TeX equivalent is in PDFTitle below
%\title{Summary of the searches for direct electroweak chargino and neutralino production with three or more leptons using 9.2 $\fbinv$ of $\sqrt{s}$ = 8 TeV CMS data}
\title{Background Combination Procedure for Multilepton Channels in SUS-12-022}

% >> Authors
%Author is always "The CMS Collaboration" for PAS and papers, so author, etc, below will be ignored in those cases
%For multiple affiliations, create an address entry for the combination
%To mark authors as primary, use the \author* form
\address[UF]{University of Florida}
\address[RUT]{Rutgers University}
%\address[cern]{CERN}
%\author[cern]{The CMS Collaboration}
\author[UF]{Lesya Shchutska}
\author[RUT]{Matthew Walker}

% >> Date
% The date is in yyyy/mm/dd format. Today has been
% redefined to match, but if the date needs to be fixed, please write it in this fashion.
% For papers and PAS, \today is taken as the date the head file (this one) was last modified according to svn: see the RCS Id string above.
% For the final version it is best to "touch" the head file to make sure it has the latest date.
\date{\today}

% >> Abstract
% Abstract processing:
% 1. **DO NOT use \include or \input** to include the abstract: our abstract extractor will not search through other files than this one.
% 2. **DO NOT use %**                  to comment out sections of the abstract: the extractor will still grab those lines (and they won't be comments any longer!).
% 3. **DO NOT use tex macros**         in the abstract: External TeX parsers used on the abstract don't understand them.
\abstract{
We present the procedure used for the combination of background estimates in multilepton channels in the EWKino search. Because multiple groups are providing results in three and four lepton channels using a variety of different techniques, different methods are used to combine different types of backgrounds in each channel. The methods by which the background estimates are obtained are discussed elsewhere.
}

% >> PDF Metadata
% Do not comment out the following hypersetup lines (metadata). They will disappear in NODRAFT mode and are needed by CDS.
% Also: make sure that the values of the metadata items are sensible and are in plain text:
% (1) no TeX! -- for \sqrt{s} use sqrt(s) -- this will show with extra quote marks in the draft version but is okay).
% (2) no %.
% (3) No curly braces {}.
\hypersetup{%
pdfauthor={Lesya Shchutska and Matthew Walker},%
%pdftitle={Summary of the searches for direct electroweak chargino and neutralino production with three or more leptons using 9.2 fb^-1 of sqrt{s} = 8 TeV CMS data},%
pdftitle={Background Combination Procedure for Multilepton Channels in SUS-12-022},%
pdfsubject={CMS},%
pdfkeywords={CMS, physics, supersymmetry}
}

\maketitle %maketitle comes after all the front information has been supplied
% >> Text
%%%%%%%%%%%%%%%%%%%%%%%%%%%%%%%%  Begin text %%%%%%%%%%%%%%%%%%%%%%%%%%%%%
%% **DO NOT REMOVE THE BIBLIOGRAPHY** which is located before the appendix.
%% You can take the text between here and the bibiliography as an example which you should replace with the actual text of your document.
%% If you include other TeX files, be sure to use "\input{filename}" rather than "\input filename".
%% The latter works for you, but our parser looks for the braces and will break when uploading the document.
%%%%%%%%%%%%%%%
\section{Introduction}
The search for direct electroweak production of SUSY particles in multilepton modes \cite{CMS-PAS-SUS-12-022} combines search regions with two or more leptons with different requirements on $\MET$, $M_{\rm T}$, dijet mass, lepton pair mass, and others to maximize sensitivity to different types of electroweak production. In the three and four lepton channels, multiple groups have designed and validated a variety of methods to estimate and understand the various contributions to background models. These methods are discussed elsewhere \cite{AN2012:248,AN2012:255,AN2012:256,AN2012:257}. In this note, we restrict ourselves to discussing the procedure for the combination of these estimates in a way that maximizes consistency. We provide justification for assumptions and choices made to accomplish this goal.

\section{Rare Processes}
The background estimates for rare processes are taken from the simulation samples described in Table \ref{tab:rare} by all three groups. Therefore, the estimates from these sources are synchronized and the resulting contributions are used.

\begin{table}
\small{
\begin{center}
\caption{\label{tab:rare}Processes contributing to rare backgrounds estimate}
\begin{tabular}{|c|}
\hline
WZJetsTo2L2Q\_TuneZ2star\_8TeV-madgraph-tauola\_Summer12\_DR53X-PU\_S10\_START53\_V7A-v1\\
TTGJets\_8TeV-madgraph\_Summer12\_DR53X-PU\_S10\_START53\_V7A-v1\\
TTZJets\_8TeV-madgraph\_v2\_Summer12\_DR53X-PU\_S10\_START53\_V7A-v1\\
TTWJets\_8TeV-madgraph\_Summer12\_DR53X-PU\_S10\_START53\_V7A-v1\\
WWJetsTo2L2Nu\_TuneZ2star\_8TeV-madgraph-tauola\_Summer12\_DR53X-PU\_S10\_START53\_V7A-v1\\
WWGJets\_8TeV-madgraph\_Summer12\_DR53X-PU\_S10\_START53\_V7A-v1\\
TTWWJets\_8TeV-madgraph\_Summer12\_DR53X-PU\_S10\_START53\_V7A-v1\\
ZZJetsTo2L2Q\_TuneZ2star\_8TeV-madgraph-tauola\_Summer12\_DR53X-PU\_S10\_START53\_V7A-v1\\
ZZJetsTo2L2Nu\_TuneZ2star\_8TeV-madgraph-tauola\_Summer12-PU\_S7\_START52\_V9-v3\\
WWZNoGstarJets\_8TeV-madgraph\_Summer12\_DR53X-PU\_S10\_START53\_V7A-v1\\
ZZZNoGstarJets\_8TeV-madgraph\_Summer12\_DR53X-PU\_S10\_START53\_V7A-v1\\
WWWJets\_8TeV-madgraph\_Summer12\_DR53X-PU\_S10\_START53\_V7A-v1\\
\hline
\end{tabular}
\end{center}
}
\end{table}

\section{Internal Conversion and ZZ}
The authors of \cite{AN2012:248} and \cite{AN2012:256} use separate data-driven methods to estimate the internal conversion background to light leptons and an official production of ZZJets to estimate the ZZ irreducible background. The authors of \cite{AN2012:255} use a private simulation sample that combines the ZZ irreducible background with the internal conversion background. We take the data-driven estimate described in \cite{AN2012:256,AN2012:257} and a synchronized estimate to account for the ZZ irreducible background.

\section{WZ}
WZ production makes up the largest irreducible background for three lepton search regions. The main source of the background estimation comes from the official WZ simulation sample. However, two of the groups apply corrections to make sure that the MET distribution is modeled properly \cite{AN2012:248,AN2012:256}. We take the average of the three estimates.

\section{Fake leptons}
The three groups have a variety of ways to estimate the backgrounds from fake leptons (both light leptons and taus). 
Because one group \cite{AN2012:248} includes the contribution from $\TTbar$ in their data-driven estimate, we average 
the data-driven contributions summed with the $\TTbar$ simulation contribution used by the two groups in the three lepton channels. 
Because only two groups \cite{AN2012:255,AN2012:256} are providing the four lepton channels, we synchronize the $\TTbar$ 
contribution for the four lepton channels and average the data-driven backgrounds.

\section{Conclusion}
%


% >> acknowledgements (for journal papers)
% Please include the latest version from https://twiki.cern.ch/twiki/bin/viewauth/CMS/Internal/PubAcknow.
%\section*{Acknowledgements}
% ack-text

%% **DO NOT REMOVE BIBLIOGRAPHY**
\bibliography{auto_generated}   % will be created by the tdr script.

%% examples of appendices. **DO NOT PUT \end{document} at the end
\clearpage


% >> acknowledgements (for journal papers)
% Please include the latest version from https://twiki.cern.ch/twiki/bin/viewauth/CMS/Internal/PubAcknow.
%\section*{Acknowledgements}
% ack-text

%%% DO NOT ADD \end{document}!