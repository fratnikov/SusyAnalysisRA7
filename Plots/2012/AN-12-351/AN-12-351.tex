% Customizable fields and text areas start with % >> below.
% Lines starting with the comment character (%) are normally removed before release outside the collaboration, but not those comments ending lines

% svn info. These are modified by svn at checkout time.
% The last version of these macros found before the maketitle will be the one on the front page,
% so only the main file is tracked.
% Do not edit by hand!
\RCS$Revision: 150335 $
\RCS$HeadURL: svn+ssh://lshchuts@svn.cern.ch/reps/tdr2/notes/AN-12-351/trunk/AN-12-351.tex $
\RCS$Id: AN-12-351.tex 150335 2012-10-03 02:44:17Z lshchuts $
%%%%%%%%%%%%% local definitions %%%%%%%%%%%%%%%%%%%%%
% This allows for switching between one column and two column (cms@external) layouts
% The widths should  be modified for your particular figures. You'll need additional copies if you have more than one standard figure size.
%\input{commands-inc.tex}
%\newcommand{\HT}{$H_TB$}
\newcommand{\fullLumi}{9.2~\textrm{fb}^{-1}}
\newcommand{\procLumi}{9.2~\textrm{fb}^{-1}}
\newcommand{\iso}{\mathrm{iso}}
\newcommand{\noniso}{\mathrm{noniso}}
\newcommand{\eiso}{\epsilon_\iso}
\newcommand{\eisomu}{\eiso(\mu)}
\newcommand{\eisotr}{\eiso(\mathrm{track})}
\newcommand{\esb}{\epsilon_\mathrm{sb}}
%\newcommand{\MET}{\ensuremath{\displaystyle{\not} E_{\rm T}}}
\newcommand{\TTbar}{\ensuremath{\rm t\bar{t}}}

\newlength\cmsFigWidth
\ifthenelse{\boolean{cms@external}}{\setlength\cmsFigWidth{0.85\columnwidth}}{\setlength\cmsFigWidth{0.4\textwidth}}
\ifthenelse{\boolean{cms@external}}{\providecommand{\cmsLeft}{top}}{\providecommand{\cmsLeft}{left}}
\ifthenelse{\boolean{cms@external}}{\providecommand{\cmsRight}{bottom}}{\providecommand{\cmsRight}{right}}
%%%%%%%%%%%%%%%  Title page %%%%%%%%%%%%%%%%%%%%%%%%
\cmsNoteHeader{12-351} % This is over-written in the CMS environment: useful as preprint no. for export versions
% >> Title: please make sure that the non-TeX equivalent is in PDFTitle below
\title{Background Combination Procedure for the searches for direct electroweak chargino and neutralino production with three or more leptons using 9.2 $\fbinv$ of $\sqrt{\text{s}}$ = 8 TeV CMS data}
%\title{Background Combination Procedure for Multilepton Channels}

% >> Authors
%Author is always "The CMS Collaboration" for PAS and papers, so author, etc, below will be ignored in those cases
%For multiple affiliations, create an address entry for the combination
%To mark authors as primary, use the \author* form
%\address[UF]{University of Florida}
%\address[RUT]{Rutgers University}
\address[cern]{CERN}
\author[cern]{SUSY Multilepton EWKino Working Group}
%\author[UF]{Lesya Shchutska}
%\author[RUT]{Matthew Walker}


% >> Date
% The date is in yyyy/mm/dd format. Today has been
% redefined to match, but if the date needs to be fixed, please write it in this fashion.
% For papers and PAS, \today is taken as the date the head file (this one) was last modified according to svn: see the RCS Id string above.
% For the final version it is best to "touch" the head file to make sure it has the latest date.
\date{\today}

% >> Abstract
% Abstract processing:
% 1. **DO NOT use \include or \input** to include the abstract: our abstract extractor will not search through other files than this one.
% 2. **DO NOT use %**                  to comment out sections of the abstract: the extractor will still grab those lines (and they won't be comments any longer!).
% 3. **DO NOT use tex macros**         in the abstract: External TeX parsers used on the abstract don't understand them.
\abstract{
We present the procedure used for the combination of background estimates in multilepton channels in the EWKino search. Because multiple groups are providing results in three and four lepton channels using a variety of different techniques, different methods are used to combine different types of backgrounds in each channel. The methods by which the background estimates are obtained are discussed elsewhere.
}

% >> PDF Metadata
% Do not comment out the following hypersetup lines (metadata). They will disappear in NODRAFT mode and are needed by CDS.
% Also: make sure that the values of the metadata items are sensible and are in plain text:
% (1) no TeX! -- for \sqrt{s} use sqrt(s) -- this will show with extra quote marks in the draft version but is okay).
% (2) no %.
% (3) No curly braces {}.
\hypersetup{%
pdfauthor={Lesya Shchutska and Matthew Walker},%
pdftitle={Background Combination Procedure for the searches for direct electroweak chargino and neutralino production with three or more leptons using 9.2 fb^(-1) of sqrt(s) = 8 TeV CMS data},%
%pdftitle={Background Combination Procedure for Multilepton Channels},%
pdfsubject={CMS},%
pdfkeywords={CMS, physics, supersymmetry}}

\maketitle %maketitle comes after all the front information has been supplied
% >> Text
%%%%%%%%%%%%%%%%%%%%%%%%%%%%%%%%  Begin text %%%%%%%%%%%%%%%%%%%%%%%%%%%%%
%% **DO NOT REMOVE THE BIBLIOGRAPHY** which is located before the appendix.
%% You can take the text between here and the bibiliography as an example which you should replace with the actual text of your document.
%% If you include other TeX files, be sure to use "\input{filename}" rather than "\input filename".
%% The latter works for you, but our parser looks for the braces and will break when uploading the document.
%%%%%%%%%%%%%%%
\section{Introduction}
The search for direct electroweak production of SUSY particles in multilepton modes \cite{CMS-PAS-SUS-12-022} combines search regions with two or more leptons with different requirements on $\MET$, $M_{\rm T}$, dijet mass, lepton pair mass, and others to maximize sensitivity to different types of electroweak production. In the three and four lepton channels, multiple groups have designed and validated a variety of methods to estimate and understand the various contributions to background models. These methods are discussed elsewhere \cite{AN2012:248,AN2012:255,AN2012:256,AN2012:257}. In this note, we restrict ourselves to discussing the procedure for the combination of these estimates in a way that maximizes consistency. We provide justification for assumptions and choices made to accomplish this goal.

\section{Rare Processes}
The background estimates for rare processes are taken from the simulation samples described in Table \ref{tab:rare} by all three groups. Therefore, the estimates from these sources are synchronized and the resulting contributions are used.

%\begin{table}
%\small
%\begin{center}
%\caption{\label{tab:rare}Processes contributing to rare backgrounds estimate}
%\begin{tabular}{|c|}
%\hline
%WZJetsTo2L2Q\!\_TuneZ2star\!\_8TeV-madgraph-tauola\!\_Summer12\!\_DR53X-PU\!\_S10\!\_START53\!\_V7A-v1\\
%TTGJets\!\_8TeV-madgraph\!\_Summer12\!\_DR53X-PU\!\_S10\!\_START53\!\_V7A-v1\\
%TTZJets\!\_8TeV-madgraph\!\_v2\!\_Summer12\!\_DR53X-PU\!\_S10\!\_START53\!\_V7A-v1\\
%TTWJets\!\_8TeV-madgraph\!\_Summer12\!\_DR53X-PU\!\_S10\!\_START53\!\_V7A-v1\\
%WWJetsTo2L2Nu\!\_TuneZ2star\!\_8TeV-madgraph-tauola\!\_Summer12\!\_DR53X-PU\!\_S10\!\_START53\!\_V7A-v1\\
%WWGJets\!\_8TeV-madgraph\!\_Summer12\!\_DR53X-PU\!\_S10\!\_START53\!\_V7A-v1\\
%TTWWJets\!\_8TeV-madgraph\!\_Summer12\!\_DR53X-PU\!\_S10\!\_START53\!\_V7A-v1\\
%ZZJetsTo2L2Q\!\_TuneZ2star\!\_8TeV-madgraph-tauola\!\_Summer12\!\_DR53X-PU\!\_S10\!\_START53\!\_V7A-v1\\
%ZZJetsTo2L2Nu\!\_TuneZ2star\!\_8TeV-madgraph-tauola\!\_Summer12-PU\!\_S7\!\_START52\!\_V9-v3\\
%WWZNoGstarJets\!\_8TeV-madgraph\!\_Summer12\!\_DR53X-PU\!\_S10\!\_START53\!\_V7A-v1\\
%ZZZNoGstarJets\!\_8TeV-madgraph\!\_Summer12\!\_DR53X-PU\!\_S10\!\_START53\!\_V7A-v1\\
%WWWJets\!\_8TeV-madgraph\!\_Summer12\!\_DR53X-PU\!\_S10\!\_START53\!\_V7A-v1\\
%\hline
%\end{tabular}
%\end{center}
%\end{table}

\begin{table}[h] 
{\footnotesize
\begin{center}
\caption{Summary of the MC samples for rare Standard Model processes. All datasets are produced with the \MADGRAPH generator. }
\label{tab:rare}
\begin{tabular}{|l|c|} \hline
  DBS Name           & $\sigma$ (pb) \\\hline \hline
/DYJetsToLL\!\_M-50\!\_TuneZ2Star\!\_8TeV-madgraph-tarball/Summer12\!\_DR53X-PU\!\_S10\!\_START53\!\_V7A-v1& 	3532.8     \\ 
/DYJetsToLL\!\_M-10To50filter\!\_8TeV-madgraph/Summer12\!\_DR53X-PU\!\_S10\!\_START53\!\_V7A-v1	& --- \\
/TTJets\!\_MassiveBinDECAY\!\_TuneZ2star\!\_8TeV-madgraph-tauola/Summer12\!\_DR53X-PU\!\_S10\!\_START53\!\_V7A-v1& 	225.2     \\ 
/TTZJets\!\_8TeV-madgraph\!\_v2/Summer12\!\_DR53X-PU\!\_S10\!\_START53\!\_V7A-v1	& 0.208      \\ 
/TTWJets\!\_8TeV-madgraph/Summer12\!\_DR53X-PU\!\_S10\!\_START53\!\_V7A-v1	& 0.232      \\ 
/TTGJets\!\_8TeV-madgraph/Summer12\!\_DR53X-PU\!\_S10\!\_START53\!\_V7A-v1	& 2.166     \\ 
/ZZZNoGstarJets\!\_8TeV-madgraph/Summer12\!\_DR53X-PU\!\_S10\!\_START53\!\_V7A-v1	& 0.01922      \\ 
/WWWJets\!\_8TeV-madgraph/Summer12\!\_DR53X-PU\!\_S10\!\_START53\!\_V7A-v1	& 0.08217      \\ 
/ZZJetsTo4L\!\_TuneZ2star\!\_8TeV-madgraph-tauola/Summer12\!\_DR53X-PU\!\_S10\!\_START53\!\_V7A-v1	& 0.1769   \\ 
/ZZJetsTo2L2Q\!\_TuneZ2star\!\_8TeV-madgraph-tauola/Summer12\!\_DR53X-PU\!\_S10\!\_START53\!\_V7A-v1	& 2.4487     \\ 
/ZZJetsTo2L2Nu\!\_TuneZ2star\!\_8TeV-madgraph-tauola/Summer12\!\_DR53X-PU\!\_S10\!\_START53\!\_V7A-v3	& 0.3648    \\ 
/WZJetsTo3LNu\!\_TuneZ2\!\_8TeV-madgraph-tauola/Summer12\!\_DR53X-PU\!\_S10\!\_START53\!\_V7A-v1	& 1.0575     \\ 
/WZJetsTo2L2Q\!\_TuneZ2star\!\_8TeV-madgraph-tauola/Summer12\!\_DR53X-PU\!\_S10\!\_START53\!\_V7A-v1	& 2.206      \\ 
/WWJetsTo2L2Nu\!\_TuneZ2star\!\_8TeV-madgraph-tauola/Summer12\!\_DR53X-PU\!\_S10\!\_START53\!\_V7A-v1	& 5.8123     \\ 
/WJetsToLNu\!\_TuneZ2Star\!\_8TeV-madgraph-tarball/Summer12\!\_DR53X-PU\!\_S10\!\_START53\!\_V7A-v1	& 37509    \\ 
/TTWWJets\!\_8TeV-madgraph/Summer12\!\_DR53X-PU\!\_S10\!\_START53\!\_V7A-v1	& 0.002      \\ 
/WWGJets\!\_8TeV-madgraph/Summer12\!\_DR53X-PU\!\_S10\!\_START53\!\_V7A-v1	& 1.44     \\ 
/WWZNoGstarJets\!\_8TeV-madgraph/Summer12\!\_DR53X-PU\!\_S10\!\_START53\!\_V7A-v1	& 0.0633      \\ 
/WZZNoGstarJets\!\_8TeV-madgraph/Summer12\!\_DR53X-PU\!\_S10\!\_START53\!\_V7A-v1 	& 0.01922      \\  \hline \hline
\end{tabular}
\end{center}
}
\end{table}


\section{Internal Conversion and ZZ}
The authors of \cite{AN2012:248} and \cite{AN2012:256} use separate data-driven methods to estimate the internal conversion background to light leptons and an official production of ZZJets to estimate the ZZ irreducible background. The authors of \cite{AN2012:255} use a private simulation sample that combines the ZZ irreducible background with the internal conversion background. We take the data-driven estimate described in \cite{AN2012:256,AN2012:257} and a synchronized estimate to account for the ZZ irreducible background.

\section{WZ}
WZ production makes up the largest irreducible background for three lepton search regions. The main source of the background estimation comes from the official WZ simulation sample. However, two of the groups apply corrections to make sure that the MET distribution is modeled properly \cite{AN2012:248,AN2012:256}. The data-driven corrections are found to be small and the final result is computed as a simple average of three predictions.

\section{Fake leptons}
The three groups have a variety of ways to estimate the backgrounds from fake leptons (both light leptons and taus). Because one group \cite{AN2012:248} includes the contribution from $\TTbar$ in their data-driven estimate, we average the data-driven contributions summed with the $\TTbar$ simulation contribution used by the two groups in the three lepton channels. Because only two groups \cite{AN2012:255,AN2012:256} are providing the four lepton channels, we synchronize the $\TTbar$ contribution for the four lepton channels and average the data-driven backgrounds.

\section{Final Tables}
The summarized results are presented in three tables: 
\begin{itemize}
\item Table~\ref{tab:OSSF1tau0} contains observed yields  and background prediction 
for each search region in a tri-lepton channel with an opposite sign same flavour lepton pair present;
\item Table~\ref{tab:OSSF0tau0}  - for a channel without  an opposite sign same flavour lepton pair;
\item Table~\ref{tab:SStau1} - for a channel with a same sign di-lepton and a hadronically decaying tau (SS$\tau$);
\item Table~\ref{tab:OSOF1tau1} - for a channel with an opposite sign opposite flavour di-lepton and a hadronically decaying tau (OSOF$\tau$). This result is fully based on the Ref.~\cite{AN2012:255}.
\end{itemize}

The graphical representation of the results is shown in Figures~\ref{fig:OSSF1tau0},~\ref{fig:OSSF0tau0},~\ref{fig:SStau1},~\ref{fig:OSOF1tau1}.

%==========================================================================================
\begin{figure}[htp]
\begin{center}
\includegraphics[width=1.0\textwidth]{plots/ossf1tau0.pdf}
\caption{Observed yields and predicted backgrounds for a tri-lepton with an opposite sign same flavour lepton pair present in all defined search regions.}
\label{fig:OSSF1tau0}
\end{center}
\end{figure}
%==========================================================================================

%==========================================================================================
\begin{figure}[htp]
\begin{center}
\includegraphics[width=1.0\textwidth]{plots/ossf0tau0.pdf}
\caption{Observed yields and predicted backgrounds for a tri-lepton without an opposite sign same flavour lepton pair present in all defined search regions.}
\label{fig:OSSF0tau0}
\end{center}
\end{figure}
%==========================================================================================

%==========================================================================================
\begin{figure}[htp]
\begin{center}
\includegraphics[width=1.0\textwidth]{plots/ossf0tau1.pdf}
\caption{Observed yields and predicted backgrounds for a tri-lepton with a same sign di-lepton and a hadronically decaying tau in all defined search regions.}
\label{fig:SStau1}
\end{center}
\end{figure}
%==========================================================================================

%==========================================================================================
\begin{figure}[htp]
\begin{center}
\includegraphics[width=1.0\textwidth]{plots/osof1tau1.pdf}
\caption{Observed yields and predicted backgrounds for a tri-lepton with an opposite sign opposite flavour di-lepton and a hadronically decaying tau  in all defined search regions.}
\label{fig:OSOF1tau1}
\end{center}
\end{figure}
%==========================================================================================

\begin{landscape}
%==========================================================================================
%==========================================================================================
\begin{table}
\begin{center}
\caption{\label{tab:OSSF1tau0} The summary of the observed yields and predicted backgrounds for tri-lepton with opposite sign same flavour pair present. }
\begin{tabular}{| c | c c c c c c c | }\hline\hline
$\ETmiss$ (GeV) & WZ & Non-Prompt & Rare SM & Z$\gamma^*$ & ZZ & Total bkg & Observed\\\hline\hline
\multicolumn{7}{l}{$M_{\text{T}} < 120$ GeV, $M_{\ell\ell} < 75$ GeV}\\\hline\hline
50$\dots$100&26$\pm$2.7&17$\pm$3.3&1.4$\pm$0.84&6.8$\pm$0.3&1.3$\pm$0.072&52$\pm$4.4&63\\
100$\dots$150&3.4$\pm$0.42&3.9$\pm$0.85&0.42$\pm$0.24&0.097$\pm$0.041&0.11$\pm$0.022&8$\pm$0.98&5\\
150$\dots$200&0.99$\pm$0.19&0.61$\pm$0.19&0.19$\pm$0.13&0.021$\pm$0.017&0.011$\pm$0.006&1.8$\pm$0.3&1\\
200$\dots$250&0.4$\pm$0.082&0.19$\pm$0.075&0.082$\pm$0.058&0.043$\pm$0.016&0.014$\pm$0.007&0.74$\pm$0.13&1\\
\hline\hline
\multicolumn{7}{l}{$120~\mathrm{GeV} < M_{\text{T}} < 160~\mathrm{GeV}$, $M_{\ell\ell} < 75$ GeV}\\\hline\hline
50$\dots$100&1.6$\pm$0.22&0.97$\pm$0.38&0.22$\pm$0.14&0.23$\pm$0.054&0.11$\pm$0.022&3.1$\pm$0.47&3\\
100$\dots$150&0.24$\pm$0.038&0.61$\pm$0.3&0.091$\pm$0.067&0.043$\pm$0.023&0.0051$\pm$0.0051&0.99$\pm$0.31&0\\
150$\dots$200&0.063$\pm$0.017&0.08$\pm$0.052&0.0046$\pm$0.0052&0$\pm$0.00049&0$\pm$0&0.15$\pm$0.055&0\\
200$\dots$250&0.043$\pm$0.015&0.075$\pm$0.037&0.044$\pm$0.044&0.022$\pm$0.018&0$\pm$0&0.18$\pm$0.062&0\\
\hline\hline
\multicolumn{7}{l}{$M_{\text{T}} > 160$ GeV, $M_{\ell\ell} < 75$ GeV}\\\hline\hline
50$\dots$100&0.8$\pm$0.28&0.77$\pm$0.47&0.16$\pm$0.19&0.13$\pm$0.04&0.096$\pm$0.021&1.9$\pm$0.58&4\\
100$\dots$150&0.6$\pm$0.21&0.75$\pm$0.54&0.096$\pm$0.11&0.022$\pm$0.016&0.058$\pm$0.015&1.5$\pm$0.59&0\\
150$\dots$200&0.16$\pm$0.063&0.61$\pm$0.48&0.11$\pm$0.13&0$\pm$0.0049&0.019$\pm$0.0093&0.9$\pm$0.5&1\\
200$\dots$250&0.18$\pm$0.07&0.064$\pm$0.042&0.033$\pm$0.038&0$\pm$0.0002&0.012$\pm$0.0048&0.28$\pm$0.09&0\\
\hline\hline
\multicolumn{7}{l}{$M_{\text{T}} < 120$ GeV, $75~\mathrm{GeV} < M_{\ell\ell} < 105~\mathrm{GeV}$}\\\hline\hline
50$\dots$100&333$\pm$33&22.6$\pm$3.7&4.1$\pm$2.2&8.2$\pm$0.3&13.1$\pm$0.2&381$\pm$33&377\\
100$\dots$150&59$\pm$6.2&2.6$\pm$0.69&1.8$\pm$1&0.14$\pm$0.049&1.6$\pm$0.083&65$\pm$6.3&61\\
150$\dots$200&15$\pm$1.9&0.49$\pm$0.14&0.64$\pm$0.35&0.022$\pm$0.019&0.29$\pm$0.035&17$\pm$1.9&13\\
200$\dots$250&8.9$\pm$1.2&0.12$\pm$0.029&0.55$\pm$0.3&0.043$\pm$0.00092&0.13$\pm$0.018&9.8$\pm$1.2&3\\
\hline\hline
\multicolumn{7}{l}{$120~\mathrm{GeV} < M_{\text{T}} < 160~\mathrm{GeV}$, $75~\mathrm{GeV} < M_{\ell\ell} < 105~\mathrm{GeV}$}\\\hline\hline
50$\dots$100&8.4$\pm$1.1&0.6$\pm$0.35&0.25$\pm$0.15&0.15$\pm$0.043&0.43$\pm$0.042&9.8$\pm$1.2&11\\
100$\dots$150&1.2$\pm$0.17&0.042$\pm$0.029&0.17$\pm$0.11&0$\pm$0.0053&0.044$\pm$0.014&1.5$\pm$0.21&0\\
150$\dots$200&0.18$\pm$0.043&0.046$\pm$0.031&0.071$\pm$0.054&0$\pm$0.00027&0$\pm$0&0.3$\pm$0.075&1\\
200$\dots$250&0.12$\pm$0.025&0.029$\pm$0.031&0.014$\pm$0.014&0$\pm$0.00025&0$\pm$0&0.17$\pm$0.042&1\\
\hline\hline
\end{tabular}
\end{center}
\end{table}
%==========================================================================================
%==========================================================================================
\begin{table*}
\begin{center}
%\caption{Continuation}
\begin{tabular}{| c | c c c c c c c | }\hline\hline
$\ETmiss$ (GeV) & WZ & Non-Prompt & Rare SM & Z$\gamma^*$ & ZZ & Total bkg & Observed\\\hline\hline
\multicolumn{7}{l}{$M_{\text{T}} > 160$ GeV, $75~\mathrm{GeV} < M_{\ell\ell} < 105~\mathrm{GeV}$}\\\hline\hline
50$\dots$100&2.3$\pm$0.72&0.34$\pm$0.14&0.24$\pm$0.27&0.24$\pm$0.054&0.17$\pm$0.026&3.2$\pm$0.78&3\\
100$\dots$150&1.5$\pm$0.49&0.078$\pm$0.047&0.15$\pm$0.17&0.065$\pm$0.028&0.036$\pm$0.013&1.8$\pm$0.53&1\\
150$\dots$200&0.48$\pm$0.17&0.015$\pm$0.016&0.12$\pm$0.14&0.043$\pm$0.023&0.015$\pm$0.0081&0.67$\pm$0.22&1\\
200$\dots$250&0.22$\pm$0.089&0.37$\pm$0.33&0.11$\pm$0.13&0$\pm$0.0001&0.01$\pm$0.0073&0.71$\pm$0.36&1\\
\hline\hline
\multicolumn{7}{l}{$M_{\text{T}} < 120$ GeV, $M_{\ell\ell} > 105$ GeV}\\\hline\hline
50$\dots$100&12$\pm$1.2&5.4$\pm$1.1&1.2$\pm$0.7&0.73$\pm$0.11&0.6$\pm$0.052&19$\pm$1.8&22\\
100$\dots$150&2.3$\pm$0.25&1.3$\pm$0.35&0.53$\pm$0.3&0.12$\pm$0.046&0.13$\pm$0.024&4.4$\pm$0.53&6\\
150$\dots$200&0.62$\pm$0.087&0.24$\pm$0.078&0.078$\pm$0.047&0.03$\pm$0.026&0.011$\pm$0.0053&0.98$\pm$0.13&2\\
200$\dots$250&0.37$\pm$0.053&0.028$\pm$0.018&0.029$\pm$0.021&0$\pm$0.00051&0.024$\pm$0.0078&0.45$\pm$0.06&2\\
\hline\hline
\multicolumn{7}{l}{$120~\mathrm{GeV} < M_{\text{T}} < 160~\mathrm{GeV}$, $M_{\ell\ell} > 105$ GeV}\\\hline\hline
50$\dots$100&0.6$\pm$0.086&0.68$\pm$0.4&0.11$\pm$0.071&0.27$\pm$0.058&0.033$\pm$0.012&1.7$\pm$0.42&0\\
100$\dots$150&0.091$\pm$0.021&0.18$\pm$0.088&0.055$\pm$0.047&0.021$\pm$0.017&0$\pm$0&0.35$\pm$0.1&2\\
150$\dots$200&0.033$\pm$0.01&0.052$\pm$0.033&0.055$\pm$0.049&0$\pm$0.0047&0$\pm$0&0.14$\pm$0.06&0\\
200$\dots$250&0.0019$\pm$0.0023&0.027$\pm$0.019&0.077$\pm$0.086&0$\pm$0.00017&0$\pm$0&0.11$\pm$0.088&0\\
\hline\hline
\multicolumn{7}{l}{$M_{\text{T}} > 160$ GeV, $M_{\ell\ell} > 105$ GeV}\\\hline\hline
50$\dots$100&0.46$\pm$0.16&0.38$\pm$0.19&0.5$\pm$0.66&0.14$\pm$0.045&0.0089$\pm$0.0053&1.5$\pm$0.71&0\\
100$\dots$150&0.21$\pm$0.081&0.22$\pm$0.059&0.49$\pm$0.62&0.19$\pm$0.05&0.014$\pm$0.0078&1.1$\pm$0.63&1\\
150$\dots$200&0.084$\pm$0.032&0.13$\pm$0.043&0.14$\pm$0.17&0$\pm$0.01&0.00068$\pm$0.00068&0.35$\pm$0.18&0\\
200$\dots$250&0.056$\pm$0.022&0.0037$\pm$8.8e-05&0.11$\pm$0.14&0.021$\pm$0.00029&0.0027$\pm$0.0027&0.19$\pm$0.14&0\\
\hline\hline
\end{tabular}
\end{center}
\end{table*}
%==========================================================================================
%==========================================================================================
\begin{table}
\small
\begin{center}
\caption{\label{tab:OSSF0tau0} The summary of the observed yields and predicted backgrounds for tri-lepton without opposite sign same flavour pair present. }
\begin{tabular}{| c | c c c c c c c | }\hline\hline
$\ETmiss$ (GeV) & WZ & Non-Prompt & Rare SM & Z$\gamma^*$ & ZZ & Total bkg & Observed\\\hline\hline
\multicolumn{7}{l}{$M_{\text{T}} < 120$ GeV, $M_{\ell\ell} < 75$ GeV}\\\hline\hline
50$\dots$100&1.6$\pm$0.16&9.4$\pm$2.1&0.44$\pm$0.26&0.92$\pm$0.092&0.13$\pm$0.023&12$\pm$2.1&12\\
100$\dots$150&0.32$\pm$0.053&2.1$\pm$0.48&0.29$\pm$0.18&0.21$\pm$0.04&0.019$\pm$0.0089&2.9$\pm$0.52&2\\
150$\dots$200&0.085$\pm$0.026&0.37$\pm$0.12&0.062$\pm$0.05&0.024$\pm$0.015&0$\pm$0&0.54$\pm$0.13&0\\
200$\dots$250&0.029$\pm$0.0081&0.099$\pm$0.026&0.13$\pm$0.14&0.032$\pm$0.015&0$\pm$0&0.29$\pm$0.14&0\\
\hline\hline
\multicolumn{7}{l}{$120~\mathrm{GeV} < M_{\text{T}} < 160~\mathrm{GeV}$, $M_{\ell\ell} < 75$ GeV}\\\hline\hline
50$\dots$100&0.29$\pm$0.042&1.3$\pm$0.43&0.28$\pm$0.22&0.1$\pm$0.029&0.027$\pm$0.011&2$\pm$0.49&1\\
100$\dots$150&0.029$\pm$0.011&0.33$\pm$0.096&0.051$\pm$0.036&0.032$\pm$0.015&0.0045$\pm$0.0045&0.45$\pm$0.1&1\\
150$\dots$200&0.019$\pm$0.0084&0.074$\pm$0.03&0.047$\pm$0.046&0$\pm$0.005&0$\pm$0&0.14$\pm$0.055&1\\
200$\dots$250&0.003$\pm$0.0038&0.058$\pm$0.038&0.0068$\pm$0.0062&0$\pm$0&0$\pm$0&0.068$\pm$0.039&0\\
\hline\hline
\multicolumn{7}{l}{$M_{\text{T}} > 160$ GeV, $M_{\ell\ell} < 75$ GeV}\\\hline\hline
50$\dots$100&0.14$\pm$0.051&0.63$\pm$0.26&0.058$\pm$0.068&0.095$\pm$0.026&0.017$\pm$0.009&0.95$\pm$0.28&1\\
100$\dots$150&0.11$\pm$0.044&0.87$\pm$0.35&0.18$\pm$0.23&0.079$\pm$0.024&0.028$\pm$0.012&1.3$\pm$0.42&0\\
150$\dots$200&0.033$\pm$0.014&0.094$\pm$0.037&0.022$\pm$0.026&0.025$\pm$0.012&0.002$\pm$0.002&0.18$\pm$0.049&0\\
200$\dots$250&0.04$\pm$0.019&0.057$\pm$0.025&0.039$\pm$0.047&0$\pm$0.00014&0.0028$\pm$0.00063&0.14$\pm$0.057&0\\
\hline\hline
\multicolumn{7}{l}{$M_{\text{T}} < 120$ GeV, $M_{\ell\ell} > 105$ GeV}\\\hline\hline
50$\dots$100&0.041$\pm$0.012&0.53$\pm$0.15&0.14$\pm$0.086&0.18$\pm$0.035&0$\pm$0&0.88$\pm$0.18&0\\
100$\dots$150&0.0064$\pm$0.0037&0.17$\pm$0.039&0.025$\pm$0.019&0.032$\pm$0.016&0$\pm$0&0.24$\pm$0.046&0\\
150$\dots$200&0$\pm$0&0.065$\pm$0.028&0.056$\pm$0.049&0.011$\pm$0.0084&0$\pm$0&0.13$\pm$0.057&0\\
200$\dots$250&0$\pm$0&0.02$\pm$0.012&0.00021$\pm$0.00017&0$\pm$0.0043&0$\pm$0&0.02$\pm$0.013&0\\
\hline\hline
\multicolumn{7}{l}{$120~\mathrm{GeV} < M_{\text{T}} < 160~\mathrm{GeV}$, $M_{\ell\ell} > 105$ GeV}\\\hline\hline
50$\dots$100&0.057$\pm$0.013&0.002$\pm$0.0015&0.023$\pm$0.019&0.021$\pm$0.011&0$\pm$0&0.1$\pm$0.025&1\\
100$\dots$150&0.0037$\pm$0.0036&0.0023$\pm$0.0016&0.014$\pm$0.014&0.022$\pm$0.011&0$\pm$0&0.042$\pm$0.018&0\\
150$\dots$200&0$\pm$0&0$\pm$0&0.00021$\pm$0.00018&0$\pm$3.4e-05&0$\pm$0&0.00021$\pm$0.00019&0\\
200$\dots$250&0$\pm$0&0$\pm$3.5e-05&0.004$\pm$0.0046&0$\pm$2.8e-05&0$\pm$0&0.004$\pm$0.0046&0\\
\hline\hline
\multicolumn{7}{l}{$M_{\text{T}} > 160$ GeV, $M_{\ell\ell} > 105$ GeV}\\\hline\hline
50$\dots$100&0.022$\pm$0.011&0.19$\pm$0.18&0.13$\pm$0.16&0.022$\pm$0.012&0.00079$\pm$0.00079&0.37$\pm$0.24&0\\
100$\dots$150&0.026$\pm$0.012&0.11$\pm$0.046&0.06$\pm$0.076&0.022$\pm$0.013&0.0044$\pm$0.0044&0.22$\pm$0.091&0\\
150$\dots$200&0.0033$\pm$0.0031&0.021$\pm$0.022&0.00061$\pm$0.00073&0$\pm$0.00011&0.0044$\pm$0.0044&0.03$\pm$0.023&0\\
200$\dots$250&0.021$\pm$0.0099&0$\pm$6.7e-05&0.044$\pm$0.058&0.011$\pm$0.0081&0$\pm$0&0.075$\pm$0.059&0\\
\hline\hline
\end{tabular}
\end{center}
\end{table}
%==========================================================================================
%==========================================================================================
\begin{table}
\small
\begin{center}
\caption{\label{tab:SStau1} The summary of the observed yields and predicted backgrounds for the channel with a same sign di-lepton and a hadronically decaying tau. }
\begin{tabular}{| c | c c c c c c c | }\hline\hline
$\ETmiss$ (GeV) & WZ & Non-Prompt & Rare SM & Z$\gamma^*$ & ZZ & Total bkg & Observed\\\hline\hline
\multicolumn{7}{l}{$M_{\text{T}} < 120$ GeV, $M_{\ell\ell} < 75$ GeV}\\\hline\hline
50$\dots$100&7.1$\pm$0.18&25$\pm$4.6&0.42$\pm$0.23&0$\pm$0&0.41$\pm$0.041&33$\pm$4.6&25\\
100$\dots$150&0.87$\pm$0.065&2.6$\pm$0.68&0.4$\pm$0.4&0$\pm$0&0.022$\pm$0.0097&3.9$\pm$0.79&0\\
150$\dots$200&0.4$\pm$0.044&0.39$\pm$0.19&0.032$\pm$0.027&0$\pm$0&0.0088$\pm$0.0059&0.83$\pm$0.2&0\\
200$\dots$250&0.21$\pm$0.032&0.071$\pm$0.056&0.023$\pm$0.017&0$\pm$0&0.0056$\pm$0.00069&0.31$\pm$0.067&0\\
\hline\hline
\multicolumn{7}{l}{$120~\mathrm{GeV} < M_{\text{T}} < 160~\mathrm{GeV}$, $M_{\ell\ell} < 75$ GeV}\\\hline\hline
50$\dots$100&1.1$\pm$0.072&1.9$\pm$0.55&0.078$\pm$0.059&0$\pm$0&0.095$\pm$0.02&3.1$\pm$0.56&3\\
100$\dots$150&0.12$\pm$0.024&0.39$\pm$0.19&0.027$\pm$0.024&0$\pm$0&0.0061$\pm$0.0044&0.54$\pm$0.2&1\\
150$\dots$200&0.02$\pm$0.0097&0$\pm$0&0.0084$\pm$0.0098&0$\pm$0&0$\pm$0&0.028$\pm$0.014&0\\
200$\dots$250&0.0054$\pm$0.0051&0.022$\pm$0.023&0.0035$\pm$0.0035&0$\pm$0&0$\pm$0&0.031$\pm$0.024&0\\
\hline\hline
\multicolumn{7}{l}{$M_{\text{T}} > 160$ GeV, $M_{\ell\ell} < 75$ GeV}\\\hline\hline
50$\dots$100&0.62$\pm$0.055&1$\pm$0.37&0.028$\pm$0.023&0$\pm$0&0.12$\pm$0.023&1.8$\pm$0.37&1\\
100$\dots$150&0.3$\pm$0.038&0.25$\pm$0.14&0.17$\pm$0.14&0$\pm$0&0.046$\pm$0.014&0.76$\pm$0.2&0\\
150$\dots$200&0.061$\pm$0.017&0.17$\pm$0.15&0.054$\pm$0.041&0$\pm$0&0.0078$\pm$0.006&0.29$\pm$0.16&0\\
200$\dots$250&0.03$\pm$0.012&0.16$\pm$0.12&0.028$\pm$0.024&0$\pm$0&0.01$\pm$0.0062&0.23$\pm$0.12&2\\
\hline\hline
\multicolumn{7}{l}{$M_{\text{T}} < 120$ GeV, $M_{\ell\ell} > 105$ GeV}\\\hline\hline
50$\dots$100&0.19$\pm$0.03&1.1$\pm$0.4&0.035$\pm$0.029&0$\pm$0&0.0074$\pm$0.0058&1.3$\pm$0.41&1\\
100$\dots$150&0.025$\pm$0.011&0.25$\pm$0.11&0.02$\pm$0.019&0$\pm$0&0$\pm$0&0.29$\pm$0.11&0\\
150$\dots$200&0.011$\pm$0.0073&0.022$\pm$0.023&0.0083$\pm$0.0082&0$\pm$0&0$\pm$0&0.042$\pm$0.026&0\\
200$\dots$250&0.0047$\pm$0.0048&0.022$\pm$0.023&0.00011$\pm$0.00011&0$\pm$0&0$\pm$0&0.027$\pm$0.024&0\\
\hline\hline
\multicolumn{7}{l}{$120~\mathrm{GeV} < M_{\text{T}} < 160~\mathrm{GeV}$, $M_{\ell\ell} > 105$ GeV}\\\hline\hline
50$\dots$100&0.035$\pm$0.013&0$\pm$0&0.02$\pm$0.018&0$\pm$0&0$\pm$0&0.055$\pm$0.023&0\\
100$\dots$150&0.013$\pm$0.0079&0$\pm$0&0.00015$\pm$0.00014&0$\pm$0&0$\pm$0&0.013$\pm$0.0079&0\\
150$\dots$200&0$\pm$0&0$\pm$0&0$\pm$0&0$\pm$0&0$\pm$0&0$\pm$0&0\\
200$\dots$250&0.0065$\pm$0.0056&0$\pm$0&0$\pm$0&0$\pm$0&0$\pm$0&0.0065$\pm$0.0056&0\\
\hline\hline
\multicolumn{7}{l}{$M_{\text{T}} > 160$ GeV, $M_{\ell\ell} > 105$ GeV}\\\hline\hline
50$\dots$100&0.034$\pm$0.013&0.18$\pm$0.12&0.0079$\pm$0.0066&0$\pm$0&0$\pm$0&0.22$\pm$0.13&1\\
100$\dots$150&0.025$\pm$0.011&0.17$\pm$0.15&0.0093$\pm$0.0086&0$\pm$0&0.0019$\pm$0.0019&0.21$\pm$0.15&1\\
150$\dots$200&0$\pm$0&0.053$\pm$0.041&0.012$\pm$0.013&0$\pm$0&0$\pm$0&0.065$\pm$0.043&0\\
200$\dots$250&0$\pm$0&0$\pm$0&0.011$\pm$0.0085&0$\pm$0&0$\pm$0&0.011$\pm$0.0085&0\\
\hline\hline
\end{tabular}
\end{center}
\end{table}
%==========================================================================================
%==========================================================================================
\begin{table}
\begin{center}
\caption{\label{tab:OSOF1tau1} The summary of the observed yields and predicted backgrounds for the channel with an OSOF di-lepton and a hadronically decaying tau. }
\begin{tabular}{| c | c c c c c c  | c  c | }\hline\hline
$\ETmiss$ (GeV) & WZ & $t\bar{t}$ & Fake tau & Z$\gamma^*$ & ZZ & Rare SM & Total bkg & Observed\\\hline\hline
\multicolumn{7}{l}{$M_{\text{T}} < 120$ GeV, $M_{\ell\ell} < 75$ GeV}\\\hline\hline
50$\dots$100&4.5$\pm$1.3&31$\pm$17&99$\pm$49&7.4$\pm$3.5&0.56$\pm$0.35&8.8$\pm$4.8&151$\pm$58&166\\
100$\dots$150&0.76$\pm$0.23&8.1$\pm$4.7&0.92$\pm$0.82&0.44$\pm$0.48&0$\pm$0&1.4$\pm$0.76&12$\pm$5.2&16\\
150$\dots$200&0.23$\pm$0.075&2.4$\pm$1.6&0$\pm$0&0$\pm$0&0$\pm$0&0.068$\pm$0.052&2.7$\pm$1.6&3\\
200$\dots$250&0.098$\pm$0.038&0.77$\pm$0.62&0$\pm$0&0$\pm$0&0$\pm$0&0.031$\pm$0.027&0.9$\pm$0.63&1\\
\hline\hline
\multicolumn{7}{l}{$120~\mathrm{GeV} < M_{\text{T}} < 160~\mathrm{GeV}$, $M_{\ell\ell} < 75$ GeV}\\\hline\hline
50$\dots$100&0.31$\pm$0.12&2.4$\pm$1.7&4.8$\pm$3.3&0.31$\pm$0.36&0.085$\pm$0.098&0.43$\pm$0.28&8.4$\pm$4.3&11\\
100$\dots$150&0.062$\pm$0.029&5.2$\pm$3.4&0$\pm$0&0$\pm$0&0$\pm$0&0.71$\pm$0.55&6$\pm$3.5&2\\
150$\dots$200&0.043$\pm$0.024&0.6$\pm$0.55&0$\pm$0&0$\pm$0&0$\pm$0&0.15$\pm$0.12&0.79$\pm$0.58&2\\
200$\dots$250&0.0057$\pm$0.006&0$\pm$0&0$\pm$0&0$\pm$0&0$\pm$0&0.022$\pm$0.024&0.027$\pm$0.025&0\\
\hline\hline
\multicolumn{7}{l}{$M_{\text{T}} > 160$ GeV, $M_{\ell\ell} < 75$ GeV}\\\hline\hline
50$\dots$100&0.11$\pm$0.12&0$\pm$0&0$\pm$0&0.44$\pm$0.65&0$\pm$0&0.089$\pm$0.11&0.64$\pm$0.81&1\\
100$\dots$150&0.11$\pm$0.12&0.64$\pm$0.95&0.61$\pm$0.9&0.51$\pm$0.74&0$\pm$0&0.75$\pm$0.92&2.6$\pm$2.9&3\\
150$\dots$200&0.025$\pm$0.028&0.72$\pm$0.95&0$\pm$0&0$\pm$0&0$\pm$0&0.15$\pm$0.18&0.9$\pm$1.1&0\\
200$\dots$250&0.016$\pm$0.019&0$\pm$0&0$\pm$0&0$\pm$0&0$\pm$0&0.097$\pm$0.12&0.11$\pm$0.13&1\\
\hline\hline
\multicolumn{7}{l}{$M_{\text{T}} < 120$ GeV, $75~\mathrm{GeV} < M_{\ell\ell} < 105~\mathrm{GeV}$}\\\hline\hline
50$\dots$100&32.4$\pm$9.2&23$\pm$13&465$\pm$141&9.0$\pm$4.5&3.1$\pm$1.7&8.6$\pm$4.7&541$\pm$152&555\\
100$\dots$150&7.5$\pm$2.1&4.5$\pm$2.8&3.9$\pm$2.5&0$\pm$0&0.65$\pm$0.41&0.96$\pm$0.53&18$\pm$5.2&16\\
150$\dots$200&2.5$\pm$0.7&0.49$\pm$0.48&0$\pm$0&0$\pm$0&0.016$\pm$0.019&0.19$\pm$0.11&3.2$\pm$0.94&4\\
200$\dots$250&1.4$\pm$0.4&0$\pm$0&0$\pm$0&0$\pm$0&0$\pm$0&0.1$\pm$0.06&1.5$\pm$0.42&2\\
\hline\hline
\multicolumn{7}{l}{$120~\mathrm{GeV} < M_{\text{T}} < 160~\mathrm{GeV}$, $75~\mathrm{GeV} < M_{\ell\ell} < 105~\mathrm{GeV}$}\\\hline\hline
50$\dots$100&0.37$\pm$0.14&2$\pm$1.4&6.7$\pm$3.6&0.44$\pm$0.45&0$\pm$0&0.52$\pm$0.33&10$\pm$4.5&12\\
100$\dots$150&0.18$\pm$0.072&2.1$\pm$1.5&0$\pm$0&0$\pm$0&0$\pm$0&0.21$\pm$0.13&2.5$\pm$1.6&4\\
150$\dots$200&0.094$\pm$0.041&0.36$\pm$0.41&0$\pm$0&0$\pm$0&0$\pm$0&0.044$\pm$0.029&0.49$\pm$0.43&0\\
200$\dots$250&0.033$\pm$0.018&0$\pm$0&0$\pm$0&0$\pm$0&0$\pm$0&0.052$\pm$0.043&0.085$\pm$0.05&0\\
\hline\hline
\end{tabular}
\end{center}
\end{table}
%==========================================================================================
\begin{table*}
\begin{center}
\begin{tabular}{| c | c c c c c c  | c  c | }\hline\hline
$\ETmiss$ (GeV) & WZ & $t\bar{t}$ & Fake tau & Z$\gamma^*$ & ZZ & Rare SM & Total bkg & Observed\\\hline\hline
\multicolumn{7}{l}{$M_{\text{T}} > 160$ GeV, $75~\mathrm{GeV} < M_{\ell\ell} < 105~\mathrm{GeV}$}\\\hline\hline
50$\dots$100&0.057$\pm$0.061&0.97$\pm$1.2&4.3$\pm$4.8&0.00044$\pm$0.00064&0$\pm$0&0.099$\pm$0.12&5.4$\pm$5.9&1\\
100$\dots$150&0.087$\pm$0.092&1.6$\pm$1.9&0$\pm$0&0$\pm$0&0$\pm$0&0.13$\pm$0.16&1.8$\pm$2.1&0\\
150$\dots$200&0.043$\pm$0.047&0$\pm$0&0$\pm$0&0$\pm$0&0$\pm$0&0.085$\pm$0.1&0.13$\pm$0.14&0\\
200$\dots$250&0.033$\pm$0.036&0$\pm$0&0$\pm$0&0$\pm$0&0$\pm$0&0.063$\pm$0.073&0.096$\pm$0.1&0\\
\hline\hline
\multicolumn{7}{l}{$M_{\text{T}} < 120$ GeV, $M_{\ell\ell} > 105$ GeV}\\\hline\hline
50$\dots$100&2.3$\pm$0.68&24$\pm$13&13$\pm$5.3&1.2$\pm$0.77&0.25$\pm$0.19&6$\pm$3.3&47$\pm$16&82\\
100$\dots$150&0.46$\pm$0.14&10$\pm$5.8&0$\pm$0&0$\pm$0&0.05$\pm$0.057&1.3$\pm$0.86&12$\pm$6&5\\
150$\dots$200&0.12$\pm$0.042&1.7$\pm$1.2&0$\pm$0&0$\pm$0&0$\pm$0&0.13$\pm$0.093&1.9$\pm$1.2&2\\
200$\dots$250&0.067$\pm$0.028&0$\pm$0&0$\pm$0&0$\pm$0&0$\pm$0&0.066$\pm$0.046&0.13$\pm$0.057&1\\
\hline\hline
\multicolumn{7}{l}{$120~\mathrm{GeV} < M_{\text{T}} < 160~\mathrm{GeV}$, $M_{\ell\ell} > 105$ GeV}\\\hline\hline
50$\dots$100&0.13$\pm$0.054&0.52$\pm$0.5&0$\pm$0&0$\pm$0&0$\pm$0&0.24$\pm$0.16&0.89$\pm$0.57&5\\
100$\dots$150&0.052$\pm$0.025&2.5$\pm$1.8&0$\pm$0&0$\pm$0&0$\pm$0&0.6$\pm$0.39&3.2$\pm$1.9&1\\
150$\dots$200&0.0087$\pm$0.0069&0$\pm$0&0$\pm$0&0$\pm$0&0$\pm$0&0.08$\pm$0.066&0.089$\pm$0.068&1\\
200$\dots$250&0.01$\pm$0.011&0$\pm$0&0$\pm$0&0$\pm$0&0$\pm$0&0.054$\pm$0.062&0.064$\pm$0.064&0\\
\hline\hline
\multicolumn{7}{l}{$M_{\text{T}} > 160$ GeV, $M_{\ell\ell} > 105$ GeV}\\\hline\hline
50$\dots$100&0.068$\pm$0.072&0.32$\pm$0.48&0$\pm$0&0$\pm$0&0$\pm$0&0.032$\pm$0.049&0.42$\pm$0.56&1\\
100$\dots$150&0.041$\pm$0.044&0.13$\pm$0.18&0$\pm$0&0$\pm$0&0$\pm$0&0.28$\pm$0.34&0.46$\pm$0.51&1\\
150$\dots$200&0.0096$\pm$0.012&0.35$\pm$0.52&0$\pm$0&0$\pm$0&0$\pm$0&0.047$\pm$0.067&0.4$\pm$0.56&1\\
200$\dots$250&0.041$\pm$0.046&0.35$\pm$0.53&0$\pm$0&0$\pm$0&0$\pm$0&0.091$\pm$0.11&0.48$\pm$0.63&0\\
\hline\hline
\end{tabular}
\end{center}
\end{table*}
%==========================================================================================
%==========================================================================================
\end{landscape}
\section{Final Tables for Four Lepton Analyses}
The summarized results are presented in three tables: 
\begin{itemize}
\item Table~\ref{tab:fourLeptonResults} contains observed yields  and background prediction 
for each search region in the four lepton channels
\end{itemize}

The graphical representation of the results is shown in 
Figures~\ref{fig:L4OSSF0tau0},~\ref{fig:L4OSSF0tau1},~\ref{fig:L4OSSF1offZtau0},~\ref{fig:L4OSSF1onZtau0},
~\ref{fig:L4OSSF1offZtau1},~\ref{fig:L4OSSF1onZtau1},~\ref{fig:L4OSSF2offZtau0}, and~\ref{fig:L4OSSF2onZtau0}.

%==========================================================================================
\begin{figure}[htp]
\begin{center}
\includegraphics[width=0.63\textwidth]{plots/4L_MET_dist_offZ_ossf0_tau0_note.pdf}
\caption{Observed yields and predicted backgrounds for four lepton events with no OSSF pairs and zero taus.}
\label{fig:L4OSSF0tau0}
\end{center}
\end{figure}
%==========================================================================================
%==========================================================================================
\begin{figure}[htp]
\begin{center}
\includegraphics[width=0.63\textwidth]{plots/4L_MET_dist_offZ_ossf0_tau1_note.pdf}
\caption{Observed yields and predicted backgrounds for four lepton events with no OSSF pairs and one tau.}
\label{fig:L4OSSF0tau1}
\end{center}
\end{figure}
%==========================================================================================
%==========================================================================================
\begin{figure}[htp]
\begin{center}
\includegraphics[width=0.8\textwidth]{plots/4L_MET_dist_offZ_ossf1_tau0_note.pdf}
\caption{Observed yields and predicted backgrounds for four lepton events with one OSSF pair offZ and zero taus.}
\label{fig:L4OSSF1offZtau0}
\end{center}
\end{figure}
%==========================================================================================
%==========================================================================================
\begin{figure}[htp]
\begin{center}
\includegraphics[width=0.8\textwidth]{plots/4L_MET_dist_offZ_ossf1_tau1_note.pdf}
\caption{Observed yields and predicted backgrounds for four lepton events with one OSSF pair off Z and one tau.}
\label{fig:L4OSSF1offZtau1}
\end{center}
\end{figure}
%==========================================================================================
%==========================================================================================
\begin{figure}[htp]
\begin{center}
\includegraphics[width=0.8\textwidth]{plots/4L_MET_dist_onZ_ossf1_tau0_note.pdf}
\caption{Observed yields and predicted backgrounds for four lepton events with one OSSF pair on Z and zero taus.}
\label{fig:L4OSSF1onZtau0}
\end{center}
\end{figure}
%==========================================================================================
%==========================================================================================
\begin{figure}[htp]
\begin{center}
\includegraphics[width=0.8\textwidth]{plots/4L_MET_dist_onZ_ossf1_tau1_note.pdf}
\caption{Observed yields and predicted backgrounds for four lepton events with one OSSF pair on Z and one tau.}
\label{fig:L4OSSF1onZtau1}
\end{center}
\end{figure}
%==========================================================================================
%==========================================================================================
\begin{figure}[htp]
\begin{center}
\includegraphics[width=0.8\textwidth]{plots/4L_MET_dist_offZ_ossf2_tau0_note.pdf}
\caption{Observed yields and predicted backgrounds for four lepton events with two OSSF pairs offZ and zero taus.}
\label{fig:L4OSSF2offZtau0}
\end{center}
\end{figure}
%==========================================================================================
%==========================================================================================
\begin{figure}[htp]
\begin{center}
\includegraphics[width=0.8\textwidth]{plots/4L_MET_dist_onZ_ossf2_tau0_note.pdf}
\caption{Observed yields and predicted backgrounds for four lepton events with two OSSF pairs onZ and zero taus.}
\label{fig:L4OSSF2onZtau0}
\end{center}
\end{figure}
%==========================================================================================
%==========================================================================================
\begin{figure}[h!]
\begin{center}
\includegraphics[width=0.8\textwidth]{plots/h_Four_invm4l_2OSSF.pdf}
\caption{Invariant mass of all four leptons in the events with four leptons, two OSSF pairs, and zero leptons. 
For comparison, we show the Higgs to ZZ to 4L expected signal.}
\label{fig:L4OSSF2Mass4L}
\end{center}
\end{figure}
%==========================================================================================

\newpage
\begin{table}[htp]
\begin{center}
\caption{\label{tab:fourLeptonResults} Four lepton backgrounds and observed yields. Backgrounds are given with statistical and systematic uncertainties, in that order.}
%\scriptsize
%\footnotesize
\tiny
\begin{tabular}{|c|c|c|ccccc|}\hline\hline
$\ETmiss$ (GeV) & Observed & Total Bkg &  $Z\gamma^{*}$ &  Rare SM & $\ttbar$ and Fake & WZ & ZZ  \\
\hline
DY2onZTau0\\
\hline
0--30 & 69 & 68.52 $\pm$ 1.04 $\pm$ 20.18 & 0.08 $\pm$ 0.04 $\pm$ 0.04 & 0.33 $\pm$ 0.02 $\pm$ 0.18 & 0.01 $\pm$ 0.01 $\pm$ 0.00 & 0.01 $\pm$ 0.01 $\pm$ 0.00 & 68.09 $\pm$ 1.04 $\pm$ 20.18 \\
30--50 & 9 & 9.31 $\pm$ 0.32 $\pm$ 3.08 & 0.10 $\pm$ 0.06 $\pm$ 0.05 & 0.21 $\pm$ 0.03 $\pm$ 0.12 & 0.01 $\pm$ 0.00 $\pm$ 0.00 & 0.01 $\pm$ 0.01 $\pm$ 0.00 & 8.98 $\pm$ 0.31 $\pm$ 3.08 \\
50--100 & 1 & 1.67 $\pm$ 0.14 $\pm$ 0.56 & 0.00 $\pm$ 0.00 $\pm$ 0.00 & 0.32 $\pm$ 0.04 $\pm$ 0.17 & 0.01 $\pm$ 0.01 $\pm$ 0.00 & 0.02 $\pm$ 0.01 $\pm$ 0.01 & 1.31 $\pm$ 0.13 $\pm$ 0.54 \\
$>$ 100 & 0 & 0.31 $\pm$ 0.03 $\pm$ 0.12 & 0.00 $\pm$ 0.00 $\pm$ 0.00 & 0.21 $\pm$ 0.03 $\pm$ 0.11 & 0.00 $\pm$ 0.00 $\pm$ 0.00 & 0.00 $\pm$ 0.00 $\pm$ 0.00 & 0.09 $\pm$ 0.01 $\pm$ 0.04 \\
\hline
DY1onZTau0\\
\hline
0--30 & 0 & 3.94 $\pm$ 0.32 $\pm$ 1.32 & 2.15 $\pm$ 0.26 $\pm$ 1.16 & 0.10 $\pm$ 0.03 $\pm$ 0.05 & 0.02 $\pm$ 0.01 $\pm$ 0.00 & 0.03 $\pm$ 0.01 $\pm$ 0.01 & 1.64 $\pm$ 0.19 $\pm$ 0.64 \\
30--50 & 1 & 0.71 $\pm$ 0.10 $\pm$ 0.18 & 0.07 $\pm$ 0.04 $\pm$ 0.04 & 0.07 $\pm$ 0.02 $\pm$ 0.04 & 0.02 $\pm$ 0.01 $\pm$ 0.00 & 0.03 $\pm$ 0.01 $\pm$ 0.01 & 0.51 $\pm$ 0.09 $\pm$ 0.17 \\
50--100 & 1 & 0.74 $\pm$ 0.09 $\pm$ 0.17 & 0.00 $\pm$ 0.00 $\pm$ 0.00 & 0.24 $\pm$ 0.04 $\pm$ 0.13 & 0.03 $\pm$ 0.01 $\pm$ 0.00 & 0.03 $\pm$ 0.01 $\pm$ 0.01 & 0.44 $\pm$ 0.08 $\pm$ 0.11 \\
$>$ 100 & 0 & 0.36 $\pm$ 0.04 $\pm$ 0.12 & 0.00 $\pm$ 0.00 $\pm$ 0.00 & 0.22 $\pm$ 0.04 $\pm$ 0.12 & 0.01 $\pm$ 0.01 $\pm$ 0.00 & 0.02 $\pm$ 0.01 $\pm$ 0.01 & 0.11 $\pm$ 0.02 $\pm$ 0.03 \\
\hline
DY2offZTau0\\
\hline
0--30 & 3 & 3.86 $\pm$ 0.26 $\pm$ 1.03 & 1.64 $\pm$ 0.23 $\pm$ 0.88 & 0.00 $\pm$ 0.00 $\pm$ 0.00 & 0.00 $\pm$ 0.00 $\pm$ 0.00 & 0.00 $\pm$ 0.00 $\pm$ 0.00 & 2.23 $\pm$ 0.12 $\pm$ 0.54 \\
30--50 & 0 & 0.42 $\pm$ 0.07 $\pm$ 0.15 & 0.12 $\pm$ 0.06 $\pm$ 0.06 & 0.00 $\pm$ 0.00 $\pm$ 0.00 & 0.00 $\pm$ 0.00 $\pm$ 0.00 & 0.00 $\pm$ 0.00 $\pm$ 0.00 & 0.30 $\pm$ 0.03 $\pm$ 0.13 \\
50--100 & 2 & 0.04 $\pm$ 0.01 $\pm$ 0.03 & 0.00 $\pm$ 0.00 $\pm$ 0.00 & 0.00 $\pm$ 0.00 $\pm$ 0.00 & 0.00 $\pm$ 0.00 $\pm$ 0.00 & 0.01 $\pm$ 0.00 $\pm$ 0.00 & 0.03 $\pm$ 0.00 $\pm$ 0.02 \\
$>$ 100 & 0 & 0.02 $\pm$ 0.01 $\pm$ 0.01 & 0.00 $\pm$ 0.00 $\pm$ 0.00 & 0.01 $\pm$ 0.01 $\pm$ 0.01 & 0.00 $\pm$ 0.00 $\pm$ 0.00 & 0.00 $\pm$ 0.00 $\pm$ 0.00 & 0.01 $\pm$ 0.01 $\pm$ 0.01 \\
\hline
DY1offZTau0\\
\hline
0--30 & 1 & 2.72 $\pm$ 0.28 $\pm$ 1.33 & 2.46 $\pm$ 0.27 $\pm$ 1.33 & 0.00 $\pm$ 0.00 $\pm$ 0.00 & 0.00 $\pm$ 0.00 $\pm$ 0.00 & 0.01 $\pm$ 0.01 $\pm$ 0.00 & 0.23 $\pm$ 0.08 $\pm$ 0.10 \\
30--50 & 0 & 0.23 $\pm$ 0.07 $\pm$ 0.10 & 0.18 $\pm$ 0.07 $\pm$ 0.09 & 0.01 $\pm$ 0.00 $\pm$ 0.00 & 0.01 $\pm$ 0.00 $\pm$ 0.00 & 0.01 $\pm$ 0.00 $\pm$ 0.00 & 0.04 $\pm$ 0.01 $\pm$ 0.01 \\
50--100 & 0 & 0.14 $\pm$ 0.04 $\pm$ 0.03 & 0.00 $\pm$ 0.00 $\pm$ 0.00 & 0.04 $\pm$ 0.02 $\pm$ 0.02 & 0.01 $\pm$ 0.01 $\pm$ 0.00 & 0.01 $\pm$ 0.01 $\pm$ 0.00 & 0.08 $\pm$ 0.03 $\pm$ 0.02 \\
$>$ 100 & 0 & 0.03 $\pm$ 0.01 $\pm$ 0.01 & 0.00 $\pm$ 0.00 $\pm$ 0.00 & 0.02 $\pm$ 0.01 $\pm$ 0.01 & 0.00 $\pm$ 0.00 $\pm$ 0.00 & 0.00 $\pm$ 0.00 $\pm$ 0.00 & 0.01 $\pm$ 0.01 $\pm$ 0.01 \\
\hline
DY0offZTau0\\
\hline
0--30 & 0 & 0.60 $\pm$ 0.14 $\pm$ 0.33 & 0.60 $\pm$ 0.14 $\pm$ 0.33 & 0.00 $\pm$ 0.00 $\pm$ 0.00 & 0.00 $\pm$ 0.00 $\pm$ 0.00 & 0.00 $\pm$ 0.00 $\pm$ 0.00 & 0.00 $\pm$ 0.00 $\pm$ 0.00 \\
30--50 & 0 & 0.17 $\pm$ 0.07 $\pm$ 0.09 & 0.17 $\pm$ 0.07 $\pm$ 0.09 & 0.00 $\pm$ 0.00 $\pm$ 0.00 & 0.00 $\pm$ 0.00 $\pm$ 0.00 & 0.00 $\pm$ 0.00 $\pm$ 0.00 & 0.00 $\pm$ 0.00 $\pm$ 0.00 \\
50--100 & 0 & 0.03 $\pm$ 0.02 $\pm$ 0.01 & 0.00 $\pm$ 0.00 $\pm$ 0.00 & 0.00 $\pm$ 0.00 $\pm$ 0.00 & 0.00 $\pm$ 0.00 $\pm$ 0.00 & 0.00 $\pm$ 0.00 $\pm$ 0.00 & 0.02 $\pm$ 0.02 $\pm$ 0.01 \\
$>$ 100 & 0 & 0.00 $\pm$ 0.00 $\pm$ 0.01 & 0.00 $\pm$ 0.00 $\pm$ 0.00 & 0.00 $\pm$ 0.00 $\pm$ 0.00 & 0.00 $\pm$ 0.00 $\pm$ 0.00 & 0.00 $\pm$ 0.00 $\pm$ 0.00 & 0.00 $\pm$ 0.00 $\pm$ 0.01 \\
\hline
DY1onZTau1\\
\hline
0--30 & 13 & 11.33 $\pm$ 0.51 $\pm$ 2.44 & 0.75 $\pm$ 0.34 $\pm$ 0.29 & 0.18 $\pm$ 0.04 $\pm$ 0.10 & 1.09 $\pm$ 0.24 $\pm$ 0.14 & 0.74 $\pm$ 0.05 $\pm$ 0.21 & 8.57 $\pm$ 0.29 $\pm$ 2.40 \\
30--50 & 4 & 5.50 $\pm$ 0.20 $\pm$ 0.73 & 0.00 $\pm$ 0.00 $\pm$ 0.00 & 0.06 $\pm$ 0.01 $\pm$ 0.03 & 0.98 $\pm$ 0.05 $\pm$ 0.05 & 1.00 $\pm$ 0.06 $\pm$ 0.29 & 3.45 $\pm$ 0.18 $\pm$ 0.67 \\
50--100 & 6 & 4.93 $\pm$ 0.17 $\pm$ 0.55 & 0.00 $\pm$ 0.00 $\pm$ 0.00 & 0.19 $\pm$ 0.05 $\pm$ 0.10 & 1.18 $\pm$ 0.06 $\pm$ 0.06 & 1.16 $\pm$ 0.06 $\pm$ 0.33 & 2.39 $\pm$ 0.14 $\pm$ 0.42 \\
$>$ 100 & 1 & 1.62 $\pm$ 0.09 $\pm$ 0.19 & 0.00 $\pm$ 0.00 $\pm$ 0.00 & 0.21 $\pm$ 0.04 $\pm$ 0.11 & 0.37 $\pm$ 0.03 $\pm$ 0.02 & 0.32 $\pm$ 0.03 $\pm$ 0.09 & 0.72 $\pm$ 0.07 $\pm$ 0.12 \\
\hline
DY1offZTau1\\
\hline
0--30 & 4 & 5.44 $\pm$ 1.39 $\pm$ 1.37 & 1.01 $\pm$ 0.18 $\pm$ 0.55 & 0.03 $\pm$ 0.02 $\pm$ 0.02 & 2.67 $\pm$ 1.37 $\pm$ 1.16 & 0.15 $\pm$ 0.02 $\pm$ 0.04 & 1.58 $\pm$ 0.13 $\pm$ 0.47 \\
30--50 & 2 & 1.72 $\pm$ 0.36 $\pm$ 0.30 & 0.55 $\pm$ 0.28 $\pm$ 0.17 & 0.01 $\pm$ 0.01 $\pm$ 0.01 & 0.14 $\pm$ 0.02 $\pm$ 0.01 & 0.14 $\pm$ 0.02 $\pm$ 0.04 & 0.87 $\pm$ 0.22 $\pm$ 0.24 \\
50--100 & 2 & 1.02 $\pm$ 0.15 $\pm$ 0.13 & 0.09 $\pm$ 0.11 $\pm$ 0.04 & 0.07 $\pm$ 0.03 $\pm$ 0.04 & 0.18 $\pm$ 0.02 $\pm$ 0.01 & 0.20 $\pm$ 0.02 $\pm$ 0.06 & 0.48 $\pm$ 0.09 $\pm$ 0.10 \\
$>$ 100 & 3 & 0.31 $\pm$ 0.06 $\pm$ 0.06 & 0.00 $\pm$ 0.00 $\pm$ 0.00 & 0.07 $\pm$ 0.03 $\pm$ 0.04 & 0.04 $\pm$ 0.01 $\pm$ 0.00 & 0.03 $\pm$ 0.01 $\pm$ 0.01 & 0.18 $\pm$ 0.05 $\pm$ 0.04 \\
\hline
DY0offZTau1\\
\hline
0--30 & 1 & 0.23 $\pm$ 0.07 $\pm$ 0.07 & 0.08 $\pm$ 0.05 $\pm$ 0.05 & 0.00 $\pm$ 0.00 $\pm$ 0.00 & 0.00 $\pm$ 0.00 $\pm$ 0.00 & 0.00 $\pm$ 0.00 $\pm$ 0.00 & 0.14 $\pm$ 0.06 $\pm$ 0.05 \\
30--50 & 0 & 0.13 $\pm$ 0.05 $\pm$ 0.03 & 0.02 $\pm$ 0.02 $\pm$ 0.01 & 0.04 $\pm$ 0.04 $\pm$ 0.02 & 0.00 $\pm$ 0.00 $\pm$ 0.00 & 0.00 $\pm$ 0.00 $\pm$ 0.00 & 0.06 $\pm$ 0.03 $\pm$ 0.02 \\
50--100 & 1 & 0.24 $\pm$ 0.09 $\pm$ 0.04 & 0.03 $\pm$ 0.07 $\pm$ 0.02 & 0.00 $\pm$ 0.00 $\pm$ 0.00 & 0.04 $\pm$ 0.01 $\pm$ 0.01 & 0.03 $\pm$ 0.01 $\pm$ 0.01 & 0.14 $\pm$ 0.06 $\pm$ 0.03 \\
$>$ 100 & 0 & 0.63 $\pm$ 0.32 $\pm$ 0.18 & 0.20 $\pm$ 0.18 $\pm$ 0.10 & 0.04 $\pm$ 0.03 $\pm$ 0.02 & 0.17 $\pm$ 0.17 $\pm$ 0.09 & 0.00 $\pm$ 0.00 $\pm$ 0.00 & 0.22 $\pm$ 0.20 $\pm$ 0.11 \\
\hline
\end{tabular}
\end{center}
\end{table}

\newpage
\section{Statistical Procedure}
To build a statistical model for this analysis we consider following sources of systematics
shared (100\% correlated between channels):
\begin{itemize}
\item uncertainties in luminosity, jet energy scale, trigger efficiency
\item uncertainty of backgrounds: WZ, ZZ, Z$\gamma$, $t\bar{t}$, rare processes
\item uncertainty of electron reconstruction, identification, selection efficiencies
\item uncertainty of muon reconstruction, identification, selection efficiencies
\item uncertainty of tau reconstruction, identification, selection efficiencies
\end{itemize}
Every channel also has two nuisances uncorrelated with other channels.
These uncertainties account for statistical fluctuations affecting
background estimations, and signal efficiency calculation.
As the total number of nuisances is proportional to number of channels used,
and therefore necessary computing resources rise exponentially with number
of combined channels,
we have to keep number of channels actually used for the analysis limited.
For the given point in the model parameter space we select a predefined
number (currently 10) of the most sensitive
channels. The sensitivity of the channel is defined as an expected limit
on the total model cross section obtained by using this channel only.
This analysis is a typical multi-channel counting experiment. Higgs group combination
tool, LandS \cite{lands}, is technically used to obtain limits.
We calculate ``LHC style'' CLs limit \cite{higgsCombination},
which effectively means using frequentist CLs with one-sided profiled likelyhood
test statistics.

\section{Conclusion}

\newpage
\appendix
\section{Graphical Comparison of the Background Estimations}
\label{app:graphical}
The efforts described in works~\cite{AN2012:248,AN2012:255,AN2012:256} are summarised for 
a review in the plots presented in this section. If not stated explicitly, the black color in the plots corresponds to 
a Ref.~\cite{AN2012:248}, the red color to a Ref.~\cite{AN2012:255}, and the blue color to a Ref.~\cite{AN2012:256}.
While for the tri-lepton processes without a tau three results are compared, for channels with a tau present
contribution were made by two out of three works which are presented in the plots.
%==========================================================================================
\begin{figure}[htp]
\begin{center}
\subfigure[]
{\label{fig:rare_3l}\includegraphics[width=1.0\textwidth]{plots/3lbkg/rare_ossf1tau0A.pdf}} \\
\subfigure[]
{\label{fig:rare_noOSSF}\includegraphics[width=1.0\textwidth]{plots/3lbkg/rare_ossf0tau0A.pdf}} \\
\caption{Comparison of the Rare SM contribution.}
\label{fig:rare1}
\end{center}
\end{figure}
%==========================================================================================
%==========================================================================================
\begin{figure}[htp]
\begin{center}
\subfigure[]
{\label{fig:rare_SStau}\includegraphics[width=1.0\textwidth]{plots/3lbkg/rare_ossf0tau1A.pdf}} \\
\subfigure[]
{\label{fig:rare_OStau}\includegraphics[width=1.0\textwidth]{plots/3lbkg/rare_ossf1tau1A.pdf}}
\caption{Comparison of the Rare SM contribution.}
\label{fig:rare2}
\end{center}
\end{figure}
%==========================================================================================
%==========================================================================================
%==========================================================================================
\begin{figure}[htp]
\begin{center}
\subfigure[]
{\label{fig:wz_3l}\includegraphics[width=1.0\textwidth]{plots/3lbkg/wz_ossf1tau0A.pdf}} \\
\subfigure[]
{\label{fig:wz_noOSSF}\includegraphics[width=1.0\textwidth]{plots/3lbkg/wz_ossf0tau0A.pdf}} \\
\caption{Comparison of the $WZ$ backgrounds.}
\label{fig:wz1}
\end{center}
\end{figure}
%==========================================================================================
%==========================================================================================
\begin{figure}[htp]
\begin{center}
\subfigure[]
{\label{fig:wz_SStau}\includegraphics[width=1.0\textwidth]{plots/3lbkg/wz_ossf0tau1A.pdf}} \\
\subfigure[]
{\label{fig:wz_OStau}\includegraphics[width=1.0\textwidth]{plots/3lbkg/wz_ossf1tau1A.pdf}}
\caption{Comparison of the $WZ$ backgrounds.}
\label{fig:wz2}
\end{center}
\end{figure}
%==========================================================================================
%==========================================================================================
%==========================================================================================
\begin{figure}[htp]
\begin{center}
\subfigure[Non-prompt leptons legend.]
{\label{fig:tt_leg}\includegraphics[width=0.5\textwidth]{plots/3lbkg/ttbar_legendA.pdf}} 
\subfigure[$ZZ$ and $Z\gamma^*$ legend.]
{\label{fig:zz_leg}\includegraphics[width=0.5\textwidth]{plots/3lbkg/zz_legendA.pdf}} 
\caption{Legends for the following plots.}
\label{fig:legends}
\end{center}
\end{figure}
%==========================================================================================
%==========================================================================================
%==========================================================================================
\begin{figure}[htp]
\begin{center}
\subfigure[]
{\label{fig:zz_3l}\includegraphics[width=1.0\textwidth]{plots/3lbkg/zz_ossf1tau0A.pdf}} \\
\subfigure[]
{\label{fig:zz_noOSSF}\includegraphics[width=1.0\textwidth]{plots/3lbkg/zz_ossf0tau0A.pdf}} \\
\caption{Comparison of the $ZZ$ and $Z\gamma^*$ contribution.}
\label{fig:zz1}
\end{center}
\end{figure}
%==========================================================================================
%==========================================================================================
\begin{figure}[htp]
\begin{center}
\subfigure[]
{\label{fig:zz_SStau}\includegraphics[width=1.0\textwidth]{plots/3lbkg/zz_ossf0tau1A.pdf}} \\
\subfigure[]
{\label{fig:zz_OStau}\includegraphics[width=1.0\textwidth]{plots/3lbkg/zz_ossf1tau1A.pdf}}
\caption{Comparison of the $ZZ$ and $Z\gamma^*$ contribution.}
\label{fig:zz2}
\end{center}
\end{figure}
%==========================================================================================
%==========================================================================================
\begin{figure}[htp]
\begin{center}
\subfigure[]
{\label{fig:fake_3l}\includegraphics[width=1.0\textwidth]{plots/3lbkg/ttbar_ossf1tau0A.pdf}} \\
\subfigure[]
{\label{fig:fake_noOSSF}\includegraphics[width=1.0\textwidth]{plots/3lbkg/ttbar_ossf0tau0A.pdf}} \\
\caption{Comparison of the non-prompt leptons contribution.}
\label{fig:fake1}
\end{center}
\end{figure}
%==========================================================================================
%==========================================================================================
\begin{figure}[htp]
\begin{center}
\subfigure[]
{\label{fig:fake_SStau}\includegraphics[width=1.0\textwidth]{plots/3lbkg/ttbar_ossf0tau1A.pdf}} \\
\subfigure[]
{\label{fig:fake_OStau}\includegraphics[width=1.0\textwidth]{plots/3lbkg/ttbar_ossf1tau1A.pdf}}
\caption{Comparison of the non-prompt leptons contribution.}
\label{fig:fake2}
\end{center}
\end{figure}
%==========================================================================================

%


% >> acknowledgements (for journal papers)
% Please include the latest version from https://twiki.cern.ch/twiki/bin/viewauth/CMS/Internal/PubAcknow.
%\section*{Acknowledgements}
% ack-text

%% **DO NOT REMOVE BIBLIOGRAPHY**
\bibliography{auto_generated}   % will be created by the tdr script.

%% examples of appendices. **DO NOT PUT \end{document} at the end
%\clearpage


% >> acknowledgements (for journal papers)
% Please include the latest version from https://twiki.cern.ch/twiki/bin/viewauth/CMS/Internal/PubAcknow.
%\section*{Acknowledgements}
% ack-text

%%% DO NOT ADD \end{document}!

